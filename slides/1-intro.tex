\documentclass[aspectratio=169]{beamer}
\usepackage{etex} % fixes new-dimension error
\usepackage{lmodern}
\usepackage[T1]{fontenc}

\usepackage{graphicx,amsmath}
\usepackage{stmaryrd} % cf. interleave
\usepackage{booktabs}
\usepackage{amscd}
\usepackage{multicol}
\usepackage[absolute,overlay]{textpos}
\usepackage{alltt}
\usepackage{proof}
\usepackage{subcaption}
\usepackage{tikz}
\usepackage{tikz-cd}
\usepackage[new]{old-arrows}
\usepackage[all]{xy}
\usepackage{pgfplots}
\usepackage{textcomp}

\input{macros/macros}
%-------------- template --------------------------------------------------
\input{macros/beamerconf}
%----------------------------------------------------------------------------

%------ using pstricks (rnode etc) ------------------------------------------
% \usepackage{pstricks,pst-node,pst-text,pst-3d}

% % context
% \AtBeginSection[]
% {
%     \begin{frame}
%         \frametitle{Table of Contents}
%         \tableofcontents[currentsection]
%     \end{frame}
% }
% \author[Renato Neves]{Renato Neves}
% % logos of institutions
% \titlegraphic{
%   \begin{textblock*}{5cm}(6.7cm,7.45cm)
%      \includegraphics[scale=0.06]{images/uminho.png}\hspace*{.85cm}~%
%   \end{textblock*}
%   \begin{textblock*}{5cm}(9.4cm,7.45cm)
%     \includegraphics[scale=0.50]{images/haslab.pdf}
%   \end{textblock*}
% }
% % No date
% \date{}

\begin{document}

% \setLectureBasic{Cyber-Physical Computation}
\setLecture{1}{Algorithms: Introduction}
\frame[plain]{\titlepage}


\section{Algorithms (CC4010)}

\begin{frame}[fragile]{Algorithms (CC4010)}
An \alert{\textbf{algorithm}} in CS is:

\begin{itemize}
  \item a \alert{method} for solving a (computational) problem
  \begin{itemize}
    \item given some \structure{input}
    \item must produce some \structure{output}
  \end{itemize}
  \item \alert{independent} of programming languages, computational machines, etc.
\end{itemize}

\begin{columns}[T]
\begin{column}{0.31\textwidth}
  \begin{alertblock}{Sorting Problem}
  \structure{Input:} a sequence $a_1, a_2, \ldots , a_n$
  \\
  \structure{Output:} a sorted permutation $a'_1 \leq a'_2 \leq \ldots \leq a'_n$
  \end{alertblock}
\end{column}\begin{column}{0.31\textwidth}
  \begin{exampleblock}{Instance}
  \structure{Input:} $4,1,5,3,7$
  \\
  \structure{Output:} $1,3,4,5,7$
  \end{exampleblock}
\end{column}\begin{column}{0.31\textwidth}
  \begin{block}{Algorithm}
  ~\\[-5pt]
  \begin{lstlisting}[language=C++,emph={[1]arr}]
int i, j;
for (i=1; i<n; i++)
  j = i-1;
  while (j>=0 &&
         arr[j]>arr[i])
    arr[j+1] = arr[j];
    j = j-1; 
  arr[j+1] = arr[i];
  \end{lstlisting}~\\[-20pt]
  \end{block}
\end{column}
\end{columns}

\end{frame}


% int i, j, min_idx;
% for (i = 0; i < n-1; i++) {
%   min_idx = i;
%   for (j = i+1; j < n; j++)
%     if (arr[j] < arr[min_idx])
%       min_idx = j;
%     if(min_idx != i)
%       swap(&arr[min_idx], &arr[i]);
% }


% int i, j;
% for (i = 1; i < n; i++) {
%   j = i - 1;
%   while (j >= 0 && arr[j] > arr[i]) {
%       arr[j + 1] = arr[j];
%       j = j - 1;
%   }
%   arr[j + 1] = arr[i];
% }





\section{Contents of the module}


\begin{frame}{Contents of the module }

  How well can we solve a \emph{problem}:
  \begin{itemize}
    \item is there an algorithm guaranteed to solve it in finite time? (\structure{Decidable})
    \item if so, is it really solving the problem? (\alert{Correct})
    \item if so, how well does it work in practice? (\alert{Feasible})
  \end{itemize}

  % We will study is a \emph{\textbf{good/better}} algorithm
  % \begin{itemize}
  % \item correct (do what is expected and always terminates)
  % \item fast/faster (does not take forever...)
  % \end{itemize}

  ~\\

  \frsplitt[0.44]{
  We will be \alert{\textbf{formal}}
  \begin{itemize}
    \item precisely formulate concepts
    \item proof correctness
    \item calculate how fast
    \item pen-and-paper (no tool support)
  \end{itemize}
  }{
  We will see \alert{\textbf{examples}}
  \begin{itemize}
    \item Some well known algorithms
    \item Understand how to reason about them
  \end{itemize}
  }
\end{frame}


\begin{slide}{Syllabus}
  \frsplitdiff{0.52}{0.45}{
    \begin{itemize}
      \item Algorithm Correctness
      \item Complexity: worst/best-case analysis
      \item Asymptotic analysis
      \item Recursive algorithms 
      \item Average-case and randomized algorithms
    \end{itemize}
  }{\!\!\!\!\!
    \begin{itemize}
      % \item Sorting algorithms
      \item Amortized analysis
      \item \textcolor{black!30}{Lower bounds}
      \item Data structures %Graph traversals and Dynamic programming
      \item Fundamentals of NP-completeness
      \\~
    \end{itemize}
  }
\end{slide}


\section{Logistics}

\begin{frame}{Useful information}
  Relevant class material and announcements will be posted on the website periodically

  {\centering\mybox[0.8]{\url{https://fm-labs.github.io/alg2425}}

  }

  \alert{Lecturers}
  \frsplitdiff{0.53}{0.51}{
    \begin{itemize}
      \item \textbf{José Proença}\\\url{https://jose.proenca.org}
      \item \href{mailto:jose.proenca@fc.up.pt}{jose.proenca@fc.up.pt}
      \item office hours: Thursday afternoon
    \end{itemize}
  }{
    \begin{itemize}
      \item Pedro Ribeiro\\\url{https://www.dcc.fc.up.pt/~pribeiro/}
      \item \href{mailto:pribeiro@fc.up.pt}{pribeiro@fc.up.pt}
      \item office hours: tbd
    \end{itemize}
  }
           
  \bigskip
           
  \alert{Office hours} (please send an email the day before if you wish to meet):
  % \begin{itemize}
  %   \item \emph{José Proença:} TBD %Friday afternoon
  % \end{itemize} 
\end{frame}

\begin{frame}{Assessment}
  Assessment will consist of
  \begin{itemize}
  \item \textbf{30\%} (\alert{IT}) -- an individual \alert{intermediate test} in the middle of the semester ($\geq 5.5$);
  % \begin{itemize}
  %   \item \textbf{10 Nov 2023};
  % \end{itemize}
  \item \textbf{70\%} (\alert{FT}) -- a \alert{final test} at the end, during the normal exam period ($\geq 5.5$);
  \item \textbf{100\%} (\structure{RE}) -- a \alert{global exam} at the end, during the recovery (\emph{recurso}) exam period  ($\geq 9.5$);
  % \item \textbf{70\%} (\structure{IT2}) -- an \structure{improvement test} that can replace the final exam (if taken);
  \end{itemize}

  There will be possible evaluation periods:
  \begin{itemize}
    \item Normal period:
      \mycbox[0.6]{$\alert{IT}*0.3 + \alert{FT}*0.7 ~~ (\geq 9.5)$}
    \item Recovery period (\emph{recurso}): 
      \mycbox[0.6]{\structure{RE} ~~ ($\geq 9.5)$}
  \end{itemize}
\end{frame}

% \byothers{Pedro Ribeiro}{4-5}{ribeiro/0_introduction_18092018.pdf}
\end{document}
